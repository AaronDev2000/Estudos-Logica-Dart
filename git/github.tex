Para subir o repositório ao GitHub, primeiro é necessário criar um repositório vazio no GitHub:
 - Indo ao seu perfil do GitHub, vá no ícone de "+" e selecione "New repository".
 - Coloque um nome único para o seu repositório.
 - Adicione uma descrição para explicar do que se trata o repositório.
 - Marque a opção conforme necessário: repositório público ou privado.
 - Clique em "Criar repositório".

Para subir modificações ao repositório no GitHub, seguimos os passos de commit normais:
    - `git add .`
    - `git commit -m "mensagem"`
    - `git merge` (se necessário). Caso não seja feito, será subida a branch sem ter sido mesclada com a principal (master/main).
    NOTA: É bom subir apenas a branch e depois fazer o merge para atualizar novamente.
    NOTA: Antes de fazer o merge, é uma boa prática atualizar com o `git pull origin`.
    - Para subir de fato, usamos o comando `git push origin (nome da branch)`.
    Isso irá mostrar a nova branch que foi subida.

Antes de criar a branch para iniciar o trabalho, temos que ter certeza de estar na parte
mais atualizada do projeto. Isso é feito através do seguinte comando:

   - `git pull origin`

Pull Request (PR):
   - Para trabalhar com PR, primeiro devemos configurar o repositório remoto nas "Settings".
   - Clique na parte de "Branches" e marque a opção "Add branch protection rule".
   - Marque a primeira opção, que exige que a pessoa realize uma Pull Request antes de fazer o merge.
   - Configure a quantidade de aprovações necessárias (normalmente 3).
   - Em "Padrão de nome", é sugerido colocar um `*` para representar todas as branches do projeto.
   - Clique em "Save changes".
   Nota: Ao fazer o Pull Request, o merge entre a branch que foi trabalhada e a principal será feito automaticamente
   após a aprovação dela.

Git Ignore:
   - Permite que certos arquivos não sejam enviados ao repositório remoto.
   - Basta apenas informar os arquivos a serem ignorados.
   - Para ignorar arquivos de uma mesma extensão, passe desta forma: `*.dart`.