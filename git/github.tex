Para suubir o repositorio ao github, primeiro e necesario criaum um repositorio vazio no github
 - Indo ao seu perfil do github, vai no icone de + e selecionar New repository
 - Colocar um nome unica ao seu repositorio
 - Colocar a descrição, para falar do que trata o repositorio
 - Marcar a opção conforme necesario, repositorio publico ou privado
 - Clicar em Criar repositorio

 
 Para subir modificaçãoes ao repositorio no github, seguimos os pasos de commit normal no
    - git add .
    - git commit -m "menssagem"
    - git merge (se necesario) caso não seja feito, sera subida a branch, sem ter sido mescalda com a principal master/main
    NOTA: e bom subir apenas a branch e depois fazer o merge para atualizar novamente.
    NOTA: Antes de fazer os merge, e uma boa pratica atualizar com o git pull origin.
    - Para subir de fato, usamos o comando Git push origin (nome da branch)
    isso ira mostrar a nova branch que foi subida,

Antes de criar a branch para iniciar o trablho, temos que ter certeza de estar na parte
mais atualizada do projeto isso e feito atravez do seguente comando

   - Git pull origin 

Pull Request (R/P)
   - Para trabllhar com pr, primeiro debemos configurar o repositorio remoto nas Settings 
   - Clicar na parte de branch, e marcar a opção (add branch protection rule)
   - Marcar a primeira opção, que pede para a pessoa realizar uma pull Request antes de fazer o merge
   - e configurar a quantidade de aprovações que precisa (3 normalmente)
   - em padrão de nome, e sujerido colocar um * para representar todos os brannch do projeto
   - clicar em save changes