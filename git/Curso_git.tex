// O que é o Git:
O Git é um sistema de controle de versão distribuído, que acompanha as alterações em arquivos ao longo do tempo, 
permitindo que se volte a versões anteriores, trabalhe em projetos colaborativamente e gerencie conflitos.
// O que é o GitHub:
O GitHub é uma plataforma de hospedagem de código-fonte e arquivos com controle de versão usando o Git. 
Ele permite que desenvolvedores colaborem em projetos, compartilhem código e gerenciem repositórios de forma eficiente.

Comandos do Git:

> O comando DIR: mostra todos os arquivos dentro de um diretório.
> O comando CD: (Change Directory), serve para navegar de um diretório a outro, passando o caminho como uma String com "".
> O comando git init: Inicializa o repositório. Ao iniciar esse repositório, o Git cria uma pasta oculta chamada .git/.
> O comando git add: é usado para colocar um arquivo no versionamento do Git. Esse comando encaminha as mudanças para o Stage.
    - O git add . irá adicionar todos os arquivos ao Stage.
> O comando git status: retorna o status atual do repositório. 
    - Quantos branches existem.
    - Quantos commits existem.
    - Indica se existem ou não arquivos que ainda não estão na área de Stage.
Nota: git config --global --add safe.directory (serve para tornar um repositório como seguro).

Antes de enviar as mudanças do Stage para o versionamento do Git, é necessário configurar nosso
usuário no Git, pois toda alteração precisa ter um autor. Isso é feito com os seguintes comandos:
> Nome: git config --global user.name Aaron
> E-mail: git config --global user.email aaronreizork1996@gmail.com
Agora sim está autorizado a subir as alterações ao versionamento.

> O comando git commit -m "mensagem de alteração": é o comando que envia os arquivos do Stage para a pasta .git.
Para subir novas mudanças no versionamento, é necessário passar pelo (git add .) e depois pelo (git commit -m "").

> O comando git log: retorna todas as versões que existem do projeto. Cada commit tem um identificador
único. O comando mostra também o autor, data e a mensagem do commit. 
O HEAD -> Aponta para a versão em que você está.

> O comando git checkout: permite voltar a versões anteriores passando os 6 primeiros valores do commit.
Porém, é apenas para visualizar versões anteriores, pois não é boa prática mudar algo no passado. Para alterar o futuro e retornar ao último commit (última versão), pode executar o 
comando git checkout master.

Para evitar conflitos de trabalho em equipe, o Git fornece uma forma de trabalhar em diferentes linhas do tempo.

Branch - Teoria: Todo branch deve derivar de um commit. 
Quando é criada uma branch, você está copiando o status daquele último commit.
Trabalhando nessa única branch, não há necessidade de baixar os commits que os demais programadores estão fazendo.
No final do dia, devem subir todas as modificações feitas nas branches. Isso é feito através do merge.
(É possível criar uma branch dentro de outra.)
 - Sempre que for desenvolver algo novo, deve criar uma branch a partir da principal (master).
 - Em caso de ter commits mais novos e minha branch ser antiga, posso trazer a branch principal
   para minha branch.

Branch - Prática: 
> O comando git branch (passando-nome-da-branch): cria a branch. Para começar a trabalhar nessa branch, deve usar o comando
git checkout e informar o nome da branch para navegar até ela.
    - O comando git branch, sem passar o nome da branch, mostra quantas branches existem e mostra o nome de cada uma delas.

    Já dentro, podemos iniciar as alterações. Agora vamos criar um commit na branch.

Merge: Cada merge gera um commit, e esse processo deve ser realizado na branch master/main, informando o nome da branch 
na qual você quer fazer a ação de merge.