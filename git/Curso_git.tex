// O que o Git:
O Git é um sistema de controle de versão distribuído, que acompanha as alterações em arquivos ao longo do tempo, 
permitindo que se voltem a versões anteriores, trabalhem em projetos colaborativamente e gerenciem conflitos.
// O que e o GitHub
O GitHub é uma plataforma de hospedagem de código-fonte e arquivos com controle de versão usando o Git. 
Ele permite que desenvolvedores colaborem em projetos, compartilhem código e gerenciem repositórios de forma eficiente.

Comandos do Git:

> O comando DIR: mostra todos os arquivos dentro de um diretorio
> O comando CD: (Change Directory), serve para navegar de um diretorio a outro, pasando o caminho
> O comando git init: Inicializa o repositório / Ao iniciar esse repositorio, o git cria uma pasta oculta, chamada .git/
> O comando git add:  e usando para colocar um arquivo no versionamento do git, esse comando encaimnha as mudanças para o Stage
    - o git add. ira adicionar todos os arquivos ao Stage 
> O comando git status: retorna o status atual do repositorio. 
    - Quantos breanch existes
    - Quantos comists existem
    - fala se existem ou não, arquivos que ainda não estão na area de Stage
Nota: git config --global --add safe.directory (serve para tornar um repositorio como seguro)

Antes de enviar as mudanças do Stage para o versionamento do git e necesario configurar nosso
usuario no Git, pois toda alteração precisa ter um autor, isso e feito no seguinte comando
> nome : git config --global user.name Aaron
> email : git config --global user.email aaronreizork1996@gmail.com
Agora sim esta autorizado a subir as alteração ao versionamento

> O comando git commit -m "mensagem de alteração": seria  o comando que envio os arquviso do Stage para a pasta .git
Para subir novas mudanças no versionamento, e necessario passar pelo (git add .) e depois pelo (git commit -m "")

> O comando git log: retorna todas as versoões que existem do projeto, cada commit tem um identificador
unico, o comando mostra tambem o autor, data, e a mensagem do commit 
O HEAD -> Aponta para a versão na aqual em que eu estou 

> Comando git checkout : permite volver a versoões anteriores pasando os 6 primeiros valores do commit

 